%______________________________________________________________________________________________________________________
% @brief    LaTeX2e Resume for Kamil K Wojcicki
\documentclass[margin,line]{templates/resume}
\usepackage{color,hyperref}
%\usepackage{color,soul}
\usepackage{comment} 
\usepackage{paralist}
\usepackage{enumitem}
%\usepackage{fancyhdr}
\usepackage[usenames,dvipsnames]{xcolor}
\definecolor{darkblue}{rgb}{0.0,0.0,0.6}
\definecolor{verydarkblue}{rgb}{0.0,0.0,0.3}
\hypersetup{colorlinks,breaklinks,
            linkcolor=black,urlcolor=black,
            anchorcolor=black,citecolor=black}
\newcommand{\myhref}[3][blue]{\href{#2}{\color{#1}{#3}}}
\newcommand{\myhrefgray}[3][blue]{\href{#2}{\color{#1}{#3}}}
\newcommand{\compresslist}{%
\setlength{\itemsep}{3pt}%
\setlength{\parskip}{0pt}%
\setlength{\parsep}{0pt}%
}
%______________________________________________________________________________________________________________________
\begin{document}
\renewcommand{\thefootnote}{\fnsymbol{footnote}} 	
\name{\huge Esha Singh}

\begin{resume}

    \section{\mysidestyle Basic\\Information}
\textbf{Citizenship:} India  								\hfill{\href{mailto:eshasingh369@gmail.com}{\nolinkurl{eshasingh369@gmail.com}}}   \vspace{0mm}\\\vspace{0mm}%
                                                                                                                                 \hfill (+91) 7019293921   \vspace{0mm}\\\vspace{0mm}%
                                                                                                                                 \hfill \myhref[black]{https://esha-singh.github.io/}{\nolinkurl{https://esha-singh.github.io/}}
    \vspace{0mm}\\\vspace{-6.5mm}%
%\line(1,0){440}
%\cfoot{asdf}
  %__________________________________________________________________________________________________________________
    % Education
\vspace{-0.1cm}    
\section{\mysidestyle Education}
       \textbf{\href{http://www.bits-pilani.ac.in/}{Birla Institute of Technology and Science, Pilani, India}}\hfill{\textbf{Aug'14 - June'18}}
    \begin{itemize}[leftmargin=*]\compresslist
        \item[]Bachelor of Engineering (Hons.), Electronics and Communication
        \textbf\\ Graduated in top 10\%
    \end{itemize}
 \vspace{-0.2cm}
    \textbf{{St.Theresa's Convent School,Karnal ,Haryana}}\hfill{\textbf{Apr'12 -- June'14}}
    \begin{itemize}[leftmargin=*]\compresslist
        \item[]XII CBSE(Sciences)\\
        Passing Percentage: 93.4\%
    \end{itemize}

%__________________________________________________________________________________________________________________
    % Research Interests
%%%%\vspace{-0.1cm}    
%%%%\section{\mysidestyle Research Interests}

%\begin{itemize}
%\itemsep0pt \item

%\begin{itemize}
%\item {\small Primarily in the areas concerning\\ 
%%%%%\textbf{Natural Language Processing:} Question Answering; Dialogue Systems; Abstractive Summarization; Language Generation; (Sent./Doc.) Representation Learning. \\
%%%%%\textbf{Machine Learning:} Deep Structured Models; Deep Generative Models; Distributed ML; Deep Reinforcement Learning, Model Interpretability \& Adversarial Attacks.
%\item {\small Worked in the areas of Planetary Ring Dynamics, Orbital Mechanics \& Control, and Space Propulsion.}
%\end{itemize}		 
%\end{itemize}
%___________________________________________________________________________________________________________________
\vspace{-0.1cm}    
\section{\mysidestyle Technical\\ Skills}
\textbf{Languages:} \textit{Proficient}: Python, C ,C++, Java $|$ \textit{Basic}: Lua, SQL, HTML/CSS, JavaScript \\
\textbf{Toolkits:} Tensorflow, Keras, XILINX, LTspice, MATLAB, Django, Wireshark, Pytorch, Spacy, Docker, LaTeX

\vspace{-0.1cm}
\section{\mysidestyle Research Experience} 
\vspace{-0.01cm}
\begin{itemize}[leftmargin = 0.7cm]\compresslist
\item~\textbf{ValueLabs}: \href{https://www.valuelabs.com/}{Machine Learning Consultant- Strategy} \hfill{\small\textit{(Jun'18 - Present)}} \vspace{-1mm}
\begin{itemize}[leftmargin=*]\compresslist
        \item[]\small\textit{Themes: Deep Learning; Computer Vision; Entity Extraction; Machine Learning}
    \end{itemize}
  
\item~\textbf{Ericsson India Global Services}: \href{https://www.ericsson.com/ourportfolio/digital-services-solution-areas/cloud-sdn?nav=fgb_101_0363}{Intern (R\&D)}\hfill{\small\textit{(July'17 - Dec'17)}}                                                                                             
\vspace{-1mm}
\begin{itemize}[leftmargin=*]\compresslist
    \item[]\small Software Defined Networking, Networking\&Cloud Computing team \vspace{-1mm} 
        \item[]\small\textit{Themes: SDN architecture; North Bound API; REST APIs; Computer Networks}   
    \end{itemize}

\item~\textbf{Centre for Development of Imaging Technology}: \href{http://www.cdit.org/}{Student Researcher} \hfill{\small\textit{(May'15 - Nov'15)}}
\vspace{-1mm}
\begin{itemize}[leftmargin=*]\compresslist
      \item[]\small Department of Optical Image Processing\vspace{-1mm}
        \item[]\small\textit{Themes:  Image \& Morphological Processing; Image compression}
    \end{itemize}
\item~\textbf{Engineers Without Borders, India}: {\href{http://www.ewb-india.org/}{Student Volunteer}}\hfill{\small\textit{(Aug'14 - Sep'15)}}
\vspace{-1mm}
\begin{itemize}[leftmargin=*]\compresslist
       \item[]\small BITS Pilani, Student Chapter\vspace{-1mm}
        \item[]\small\textit{Themes: Optics; LED; Electronic Circuits}
    \end{itemize}
    \end{itemize}
%%%%%%%%%%%%%%WORKEXPERIECE%%%%%%%%%%%%%
\vspace{-0.1cm}
\section{\mysidestyle Recent\\Projects}
\textsf{\textbf{Customer Service Auto-Reply System}} \hfill{\myhref[darkblue]{https://esha-singh.github.io/\#mpi}{Web}}
\vspace{0.05cm}
\begin{itemize}[leftmargin=*]\compresslist
 \item[--] Key problem was to Auto-generate responses on the Customer service side, where the response corresponds to only one query category, from available 14 query classes (Order modification, cancellation etc.) 
 \item[--] First, used SVM to classify the user query in the pre-decided 14 query classes.
 \item[--] Then, used an encoder-decoder architecture for generating the response. Pre-processing steps(language detection, tokenization, sentence segmentation)+ LSTM for response selection + response generation(includes Semantic Intent clustering, Semi-supervised Learning ,Cluster Validation)
\vspace{0.05cm}  
\end{itemize}

\vspace{-0.1cm}    
\textsf{\textbf{Question Generator for Human Welfare and Health Publisher}} \hfill{\myhref[darkblue]{https://esha-singh.github.io/\#mpi}{Web}}
\vspace{0.05cm}
\begin{itemize}[leftmargin=*]\compresslist
\item[--]Problem Statement was to generate all questions possible from given journal/article/paragraph.
\item[--]Questions are generated using an RNN encoder-decoder architecture while adopting the global attention mechanism.
\item[--]Evaluation was based on BLEU and FM scores. The model was able to generate 4 lucid questions from any 5 line paragraph.
\end{itemize}

\vspace{-0.1cm}    
\textsf{\textbf{Answer Generation for Reading Comprehension}} \hfill{\myhref[darkblue]{https://esha-singh.github.io/\#mpi}{Web}}
\vspace{0.05cm}
\begin{itemize}[leftmargin=*]\compresslist
\item[--]Needed to generate answers to User-input questions. The answers should be generated, using content from the given database of articles/ journals
\item[--]Used Information Retrieval + Neural  machine comprehension model for extractive question answering.
\item[--]An article retriever model using bigram hashing and TF-IDF weighted bag-of-word vectors designed to, given a question, efficiently return a subset of relevant articles.
\item[--]A multi-layer recurrent neural network machine comprehension model trained to detect answer spans in those returned documents ( Wikipedia Articles corpus + SQuAD+ Propriety journals)
\end{itemize}

\newpage
\vspace{-0.1cm} 
\textsf{\textbf{Unstructured to Structured Data for Corporate Invoices using Machine Learning}}\hfill{\myhref[darkblue]{https://esha-singh.github.io/\#mpi}{Web}}
\vspace{0.05cm}
\begin{itemize}[leftmargin=*]\compresslist
\item[--]Aim was to extract all intelligible information from unstructured data sans any Rule-based logic. 
\item[--]Used Tesseract for OCR of invoices \& developed the algorithm using Fuzzy matching(Levenshtein, Jaro–Winkler distance).Also working on signature extraction using connected-component analysis.
\item[--]The algorithm is independent of invoice formats with a Precision of 83\% (over 2k documents)(Precision-Recall metric)
\item[--]This project was in collaboration with a major American auto-finance company and estimated revenue generation projected is \$800 per 10 invoices per month.

\end{itemize}

\vspace{-0.1cm}    
\textsf{\textbf{Vehicle Route Optimization}} \hfill{\myhref[darkblue]{https://esha-singh.github.io/\#mpi}{Web}}
\vspace{0.05cm}
\begin{itemize}[leftmargin=*]\compresslist
    \item[--]Aim was to find the most optimized route to take with constraint in the time window and load of the vehicle.
   \item[--] Worked on creating a Vehicle Route Optimization interface where we use Geo-codes(UK) to pinpoint locations. 
   \item[--]Dijkstra, A* and Contraction Hierarchies are supported by the route optimization engine.
\end{itemize}

\vspace{-0.2cm}
\textsf{\textbf{Journal/Article Summarization}} \hfill{\myhref[darkblue]{https://esha-singh.github.io/\#mpi}{Web}}
\vspace{0.05cm}
\begin{itemize}[leftmargin=*]\compresslist
 \item[--] Summarise any firms' Form 10-k documents with concise \& appropriate information. 
\item[--]An architecture which is an Extractive + Abstractive model for summarization using multilayer LSTMs.
\item[--]Abstractive : Seq-Seq Attentional model + Pointer Network + Coverage mechanism(to preempt repetitions, Abigail et.al) 
\item[--]Also, working on LDA based corporate risk factor and stock analysis on 10-k forms.
\end{itemize}

\vspace{-0.2cm}    
\textsf{\textbf{Image Segmentation (Unet )}} \hfill{\myhref[darkblue]{https://esha-singh.github.io/\#mpi}{Web}}
\vspace{0.05cm}
\begin{itemize}[leftmargin=*]\compresslist
 \item[--]Given a corpus of 4k satellite images of 1024x1024 dimensions, need to identify which segment in the image contains salt.  
\item[--] Participated in Kaggle Competition of TGS Salt Identification Challenge.
\item[--] In a team of three built a Unet (image segmentation) architecture with Lovász-hinge loss algorithm, which gave us a score of 0.82 ( IOU metric) on the provided test set.
\end{itemize}

\vspace{-0.2cm}    
\textsf{\textbf{Fall Detection device using BMA280}} \hfill{\myhref[darkblue]{https://esha-singh.github.io/\#mpi}{Web}}
\vspace{0.05cm}
\begin{itemize}[leftmargin=*]\compresslist
\item[--]Target was to engineer a fall detection system for the elderly as a part of Self-care kits.
\item[--]Worked on prototyping the system for a device which uses BMA280 \& a microprocessor.
\item[--]Sans any visual device embeddings like camera needed to develop a system which can predict free fall based on accelerometer readings, Heart-rate, blood pressure \& perspirations. We identified four basic parameters for free fall detection which effected where the device is to be worn like on wrist, waist or around the neck.
\end{itemize}

\vspace{-0.2cm}    
\textsf{\textbf{Fault/Anomaly detection among printing logos on packages}} \hfill{\myhref[darkblue]{https://esha-singh.github.io/\#mpi}{Web}}
\normalsize
\vspace{0.05cm}
\begin{itemize}[leftmargin=*]\compresslist
  \item[--] Aim was to predict 3 minutes into future when a logo printer will deviate from printing accurate logo design, using Machine Learning.
\item[--] Worked with a team of 4 to develop a system to detect fault register 3 minutes into future using LSTMs, statistical analysis and linear regressions on signal data from 1500 sensors. 
\end{itemize}

\vspace{-0.2cm}    
\textsf{\textbf{GANs for Neural Dialogue Generation}} \hfill{\myhref[darkblue]{https://esha-singh.github.io/\#mpi}{Web}}
\normalsize
\vspace{0.05cm}
\begin{itemize}[leftmargin=*]\compresslist
  \item[--]Experimented with GAN and Seq-GANs to model dialogue generation.  
\item[--] Experimented implementing J. Li, W. Monroe, T. Shi, A. Ritter, and D. Jurafsky. Adversarial learning for neural dialogue generation. arXiv preprint arXiv:1701.06547, 2017.
\end{itemize}

%\newpage
\section{\mysidestyle Past\\Projects}
\textsf{\textbf{Integrated SDN Troubleshooter Interface}}\hfill\textit{\small(July'17 - Dec'17)}
\\ {\textit{Guide: {Vivek Srivastava, Principle Engineer(SDN R\&D)}}}\hfill{\myhref[darkblue]{https://esha-singh.github.io/\#mpi}{Web}}
\vspace{0.05cm}
\begin{itemize}[leftmargin=*]\compresslist
\item[--]Project-based on development and designing of an integrated troubleshooting environment for SDN(Kafka dumps)
\item[--]The troubleshooting interface interconnects Control Data-Plane-Interface, SDN Controller and outgoing Northbound APIs. The interface has been released for scale testing.
\end{itemize}

\newpage
\vspace{-0.2cm}    
\textsf{\textbf{"Producing Veridical Facial Line Drawings From Obscure Images In MATLAB"}}\hfill\textit{\small(May'15 - Nov'15)}
\\ {\textit{Guide: \href{http://www.cdit.org/who-who}{ Sri. Sajan Ambadiyil, Head of Department Design and Research, OIP}}} \hfill{\myhref[darkblue]{https://esha-singh.github.io/\#mpi}{Web}}
\normalsize
\vspace{0.05cm}
\begin{itemize}[leftmargin=*]\compresslist
  \item[--]Wrote a thesis \& developed an algorithm for facial recognition of humans using MATLAB \& its conversion into veridal lines. The algorithm has 90\% success rate.
\item[--]Created dataset of 4k images from video footages of criminal activities. Further segregated them into 3 categories based on facial angle visibility. 
\end{itemize}

\vspace{-0.2cm}    
\textsf{\textbf{“Low-power light source using LEDs\&waste materials for villages near}}\hfill\textit{\small(Aug'14 - Sep'15)}\\
\textsf{\textbf{ BITS Pilani”}}
\normalsize
\vspace{0.05cm}
\begin{itemize}[leftmargin=*]\compresslist
    \item[--]Worked with EWB (Engineers without Borders) organization - India Chapter, as a member, in prototype designing for the project.
    \item[--]We made 100 prototypes and distributed to nearby villages for 1-5 rupees (\$0.014 - \$0.068) 
\end{itemize}

\section{\mysidestyle Thesis}
\textsf{\textit{"Particle Swarm Optimisation - Applications"}}\hfill\textit{\small(Aug '18 - present)}
\\ {\textit{Guide: \href{https://www.iitr.ac.in/centers/NT/pages/People+Faculty+Dr__Nagendra_Prasad_Pathak.html}{Prof. Nagendra Prasad Pathak, IIT- Roorkee}}} \hfill{\myhref[darkblue]{https://esha-singh.github.io/\#mpi}{Web}}
\vspace{0.05cm}
\begin{itemize}[leftmargin=*]\compresslist
\item[--]Working towards writing a research paper which will include custom Algorithm and data-set. 
\item[--]We use PSO technique to find the most optimized solution for electromagnetic device parameters and readings. Will be extrapolated to neural networks and related use-cases. 
\end{itemize}

\vspace{-0.2cm}    
\textsf{\textit{“Lab-On-Chip PCR Miniaturization using MEMs For Molecular Diagnosis \& Cure Generation of MDR-TB”}}\hfill\textit{\small(Aug '16 - Jan '17)}
\\ {\textit{Guide: \href{http://www.bits-pilani.ac.in/hyderabad/sgoel/profile}{ Prof. Sanket Goel, Associate Professor \& Head, Department of EEE, BITS Pilani}}}\hfill{\myhref[darkblue]{https://esha-singh.github.io/\#mpi}{Web}}
\normalsize
\vspace{0.05cm}
\begin{itemize}[leftmargin=*]\compresslist
    \item[--] Worked extensively with  Peltier thermostats, thermocyclers, and  PID controllers to design a miniature on-chip PCR.
   
\end{itemize}

\section{\mysidestyle Relevant Coursework} 
Machine Learning, Data Structures, Computer Architecture, Operating Systems, Object Oriented Programming, Algorithms,  Informational Retrieval, Database Systems, Digital Design, Microprocessors, Advanced Calculus, Discrete Mathematics, Optimisation, Computer Networks, Probability \& Statistics, Neural Networks

% \vspace{-0.2cm}    
% \section{\mysidestyle Test Scores}
% \textbf{GRE:} 330/340  $|$  \textbf{TOEFL:} 111/120



%_________________________________________________________________________________________________________________________________
\vspace{-0.2cm}    
\section{\mysidestyle Positions of \\ Responsibility}
  \textbf{\href{https://www.valuelabs.com/}{ValueLabs, Hyderabad, India}}\hfill{\textit{Aug'18 -- Present}}
\begin{itemize}[leftmargin=*]\compresslist
\item[--]Panel Member for All India recruitment drive for Technical consultants Level 1 for IIT Madras, IIIT Banglore, IIIT Allahabad, CBIT, SRM University, and NIT Warangal.
\item[--]Represented Valuelabs, India for INSPIRE 2018, London
\end{itemize}
\vspace{-0.2cm}
  \textbf{Placement Logistic Coordinator,BITS}\hfill{\textit{Sep'14 - Dec'17}}
\begin{itemize}[leftmargin=*]\compresslist
\item[]Hosted 3 companies for campus placements: Accenture, Microsoft, MuSigma as a coordinator.
\end{itemize}
\vspace{-0.2cm}
  \textbf{Nirmaan, Student Volunteer}\hfill{\textit{Oct'14 - Dec'17}}
\begin{itemize}[leftmargin=*]\compresslist
\item[]Organised various events for the organisation. Also, taught primary students of villages around the college Arithmetic \&  English.
\end{itemize}
\vspace{-0.2cm}
  \textbf{Student Prefect Leader}\hfill{\textit{Aug'13 - Mar'14}}
 \begin{itemize}[leftmargin=*]\compresslist
\item[]St. Theresa's Convent School
\end{itemize}
\vspace{-0.2cm}
  \textbf{Haryana State Skating Team Captain}\hfill{\textit{Jun'02 - Dec'11}}
 \begin{itemize}[leftmargin=*]\compresslist
\item[]Captain of Haryana Girls Skating Team which competed on state and National level
\end{itemize}
%__________________________________________________________________________________________________________________________________

\vspace{-0.1cm}    
\section{\mysidestyle Achievements\\\& Awards} 
\vspace{0.01cm}
\begin{itemize}[leftmargin=*]\compresslist
\item Recipient of Merit Scholarship - BITS PILANI 
\item Qualified All India Physics Teachers' Association Exam (A Homi Bhabha Science Education Exam) with 80 percentile in 2012
\item Qualified Science Olympiad with AIR 164, Math Olympiad with a state rank of 1 (2009, 2010,2011, 2012, 2013, 2014)
\item National level skater: state champion ( Haryana) 4 times
\item Avid debater with state-level prizes ( 1st, 2nd) throughout schooling
\end{itemize}
%___________________________________________________________________________________________________________________________________

\end{resume}
\end{document}

%______________________________________________________________________________________________________________________
% EOF
